\section{Properties\label{sec:thm}}

\subsection{Type Safety}

We have proven type safety of the calculus following the syntactic approach of \citet{DBLP:journals/iandc/WrightF94}.

\begin{theorem}[Preservation]
    \coq{preservation}\\
    If $\typingG{N}{A}$ and $\red{N}{M}$, then
    either $M = \blame{p}$ for some $p$ or $\typingG{M}{A}$.
\end{theorem}
\begin{proof}
    The proof is by induction on the typing derivation.
    The \rn{App} and \rn{Case-NonNull} cases depend on a substitution lemma, shown below.
    The \rn{Cast-App} case depends on symmetry of the compatibility relation $\sim$.
\end{proof}

\begin{lemma}[Substitution]
    \coq{substitution}\\
    If $\typing{\Gamma,x:B}{N}{A}$ and $\typingG{M}{B}$, then $\typingG{\subst{N}{x}{M}}{A}$.
\end{lemma}
\begin{proof}
    The proof is standard, by induction on the typing of $N$. The \rn{Var} case depends on
    a weakening lemma, also proved by straightforward induction.
\end{proof}

\begin{lemma}[Compatibility Symmetry]
    \coq{compat\_sym}
    If $\compat{A}{B}$ then $\compat{B}{A}$.
\end{lemma}
\begin{proof}
    The proof is by straightforward induction on the derivation of $\compat{A}{B}$.
\end{proof}

\begin{theorem}[Progress]
    \coq{progress}
    If $\typing{}{N}{A}$ then either $N$ is a value,
    $\red{N}{M}$ for some $M$, or $N = \blame{p}$ for some $p$.
\end{theorem}
\begin{proof}
    The proof is by induction on the typing derivation.
    The \rn{App} case depends on a canonical forms lemma for function types, shown below.
\end{proof}

\begin{lemma}[Canonical Forms Arrow]
    \coq{canonical\_forms\_arrow}\\
    If $\typing{}{V}{\arrow{A}{B}}$, then
    either $V = \abs{x}{A}{N}$ for some $x$ and $N$, or
    $V = \cast{W}{\arrow{A'}{B'}}{p}{\arrow{A''}{B''}}$ for some
    $W, A', B', A'', B''$, and $p$.
\end{lemma}
\begin{proof}
    The proof is by straightforward induction on the typing derivation.
\end{proof}

\subsection{Blame Safety}

In addition to type safety, we have also proved blame safety following directly
the approach of \citet{DBLP:conf/esop/WadlerF09} (see also \citet{DBLP:conf/snapl/Wadler15} for a more accessible summary of the approach). The applicability of this standard
approach is one of the benefits of the calculus relative to the calculus of
\citet{DBLP:conf/ecoop/NietoRRL19}.

The approach depends on four subtyping relations, defined for the
calculus in \Cref{fig:subtyping}.  Intuitively, a cast between types
related by positive subtyping cannot give rise to positive blame
(blame with the same label as the cast) and a cast between types
related by negative subtyping cannot give rise to negative blame
(blame with a label that is the negative of that on the
cast). Ordinary subtyping is an intersection of these two
relations, so a cast between types related by ordinary subtyping
cannot give rise to any blame.  We will discuss naive subtyping and
its relationship to the other three subtyping relations in
\Cref{sec:tangram}.

\begin{figure}
    \infax[Base]
    {\subtyp{\iota}{\iota}}

    \infrule[Null-Sup]
    {\subtyp{D}{E}}
    {\subtyp{D}{E?}}

    \infrule[Null]
    {\subtyp{D}{E}}
    {\subtyp{D?}{E?}}

    \infrule[Arrow]
    {\subtyp{A'}{A} \andalso \subtyp{B}{B'}}
    {\subtyp{\arrow{A}{B}}{\arrow{A'}{B'}}}

    \infax[Naive-Base]
    {\subtypn{\iota}{\iota}}

    \infrule[Naive-Null-Sup]
    {\subtypn{D}{E}}
    {\subtypn{D}{E?}}

    \infrule[Naive-Null]
    {\subtypn{D}{E}}
    {\subtypn{D?}{E?}}

    \infrule[Naive-Arrow]
    {\shaded{\subtypn{A}{A'}} \andalso \subtypn{B}{B'}}
    {\subtypn{\arrow{A}{B}}{\arrow{A'}{B'}}}

    \infax[Positive-Base]
    {\subtyppos{\iota}{\iota}}

    \infrule[Positive-Null-Sup]
    {\subtyppos{D}{E}}
    {\subtyppos{D}{E?}}

    \infrule[Positive-Null]
    {\subtyppos{D}{E}}
    {\subtyppos{D?}{E?}}

    \infrule[Positive-Arrow]
    {\shaded{\subtypneg{A'}{A}} \andalso \subtyppos{B}{B'}}
    {\subtyppos{\arrow{A}{B}}{\arrow{A'}{B'}}}

    \infax[Negative-Base]
    {\subtypneg{\iota}{\iota}}

    \infrule[Negative-Null-Sup]
    {\subtypneg{D}{E}}
    {\subtypneg{D}{E?}}

    \infrule[Negative-Null]
    {\subtypneg{D}{E}}
    {\subtypneg{D?}{E?}}

    \infrule[Negative-Arrow]
    {\shaded{\subtyppos{A'}{A}} \andalso \subtypneg{B}{B'}}
    {\subtypneg{\arrow{A}{B}}{\arrow{A'}{B'}}}

  \caption{Subtyping rules of null blame calculus}\label{fig:subtyping}
\end{figure}



We make the intuitive understanding of positive and negative subtyping precise as follows:
\begin{definition}[Safe Term\label{def:sfor}]
    \coq{sfor}
    A term $N$ is \emph{safe} for blame label $p$, written
    $\sfor{N}{p}$, if $N$ has no subterm of the form $\blame{p}$,
    every cast in $N$ of the form $\cast{M}{A}{p}{B}$ satisfies $\subtyppos{A}{B}$,
    and every cast in $N$ of the form $\cast{M}{A}{\overline{p}}{B}$
    satisfies $\subtypneg{A}{B}$.
\end{definition}

With this definition, we can prove blame safety of the calculus, that
when $\sfor{N}{p}$, $N$ cannot reduce to $\blame{p}$ in any number of
steps.

\begin{theorem}[Safe Term Preservation]
    \coq{sfor\_preservation}\\
    If $\typingG{N}{A}$, $\sfor{N}{p}$, and $\red{N}{M}$, then $\sfor{M}{p}$.
\end{theorem}
\begin{proof}
    The proof is by induction on the derivation of $\red{N}{M}$.
    Each case is straightforward except in the \rn{App} and \rn{Case-NonNull} cases,
    we need the following lemma to show that the safety relation is preserved by
    substitution.
\end{proof}

\begin{lemma}[Substitution Preserves Safe Terms]
    \coq{subst\_pres\_sfor}\\
    If $\sfor{N}{p}$ and $\sfor{M}{p}$, then $\sfor{\subst{N}{x}{M}}{p}$.
\end{lemma}
\begin{proof}
    By straightforward induction on the structure of $N$.
\end{proof}

\begin{theorem}[Safe Term Progress]
    \coq{sfor\_progress}\\
    If $\typing{}{N}{A}$, $\sfor{N}{p}$, and $\red{N}{\blame{q}}$, then $p \neq q$.
\end{theorem}
\begin{proof}
    By straightforward induction on the derivation of $\red{N}{\blame{q}}$.
\end{proof}

\subsection{Naive Subtyping and the Tangram Lemma\label{sec:tangram}}

Naive subtyping relates types according to how \emph{definite} they are
in the sense of gradual typing. In our specific setting, $A$ is a naive
subtype of $B$ if they have the same structure, but some non-nullable components $D$ of $A$
may be replaced by nullable components $D?$ in $B$, regardless of whether they occur
covariantly or contravariantly. In a language with implicit nulls, the less definite
type $D?$ appearing in a program might be intended to mean either a non-nullable
type $D$ or a nullable type $D?$; the distinction would be expressible using these
more definite types in a language with explicit nulls.

\TODO{I'm not entirely sure about the following paragraph.}
\TODO{PLW: I suggest deleting the paragraph, and leaving this
  discussion for the section on implicit and explicit languages}
Naive subtyping
corresponds to the situation when types from a language with explicit nulls (such
as Scala) are approximated using types in a language with implicit nulls (such as Java),
or conversely, when a compiler for a language with explicit nulls needs to understand
types that it reads from code in a language with implicit nulls.
The implementation of the Scala 3 compiler uses naive subtyping to translate types
from Java code into its own Scala internal representation.

The Tangram Lemma of \citet{DBLP:conf/esop/WadlerF09} relates these four subtyping
relations. We show that it holds for the specific relations defined in \Cref{fig:subtyping}.

\begin{theorem}[Tangram Lemma\label{thm:tangram}]
    \coq{tangram\_fwd}
    \coq{tangram\_rev}
    \coq{tangram\_naive\_fwd}
    \coq{tangram\_naive\_rev}
    \begin{enumerate}
        \item $\subtyp{A}{B}$ if and only if $\subtyppos{A}{B}$ and $\subtypneg{A}{B}$.
        \item $\subtypn{A}{B}$ if and only if $\subtyppos{A}{B}$ and $\subtypneg{B}{A}$.
    \end{enumerate}
\end{theorem}
\begin{proof}
    Each of the four cases is proved by a straightforward induction on the derivation
    of $\subtyp{A}{B}$, the derivation of $\subtypn{A}{B}$, or mutual induction
    on the derivations of $\subtyppos{A}{B}$ and $\subtypneg{A}{B}$.
\end{proof}

\TODO{Explain the practical consequence of the Tangram Lemma for the Scala/Java case.}

\TODO{PLW: Explain the grayed out portions of Figure 4, or remove them.}