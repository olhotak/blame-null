\section{An Implicitly Nullable Language\label{sec:impdefn}}

Having defined a core calculus with casts, we now extend it to
model the interaction of two languages, one with explicit nulls (like Scala)
and one with implicit nulls (like Java). To do this, \citet{DBLP:conf/ecoop/NietoRRL19}
defined two new calculi with translations to a third core calculus.
Instead, we reuse our core calculus directly as the language with
explicit nulls, and define a more restricted form of it as the language
with implicit nulls.

\subsection{Syntax}

The syntax of the two-language calculus is defined in \Cref{fig:impsyntax}.
The terms $L, M, N$ of the explicit sublanguage are just those of the core calculus
from \Cref{fig:syntax}, extended with $\itrm{\hat{M}}$, 
an embedding of a term $\hat{M}$ of the implicit sublanguage within a term
of the explicit sublanguage. In general, we use a $\hat{\text{hat}}$ to mark elements of the implicit sublanguage.

\begin{figure}
  \textbf{Implicit Terms}\\
  \begin{align*}
      \hat{L}, \hat{M}, \hat{N} &\Coloneqq \ \hat{x}    \tag*{Variable} \\
      &\p \hat{c}                     \tag*{Base Constant} \\
&\p \binop{\hat{M}}{\hat{N}}          \tag*{Base Operation} \\
&\p \hat{\code{null}}           \tag*{Null Constant} \\
&\p \abs{x}{\hat{A}}{\hat{N}}         \tag*{Function Abstraction} \\
&\p \app{\hat{L}}{\hat{M}}            \tag*{Function Application} \\
  \end{align*}
  \textbf{Implicit Types}\\
  \begin{align*}
      \hat{A}, \hat{B}, \hat{C} &\Coloneqq \iota       \tag*{Base Type} \\
      &\p \arrow{\hat{A}}{\hat{B}}          \tag*{Function Type} \\
  \end{align*}
  \caption{Syntax of the language with implicit nulls}\label{fig:impsyntax}
\end{figure}


The syntax of terms of the implicit sublanguage $\hat{L}, \hat{M}, \hat{N}$
mirrors the core calculus, but omits lifting, pattern matching, casts, and blame,
since those are useless without the distinction between nullable and non-null
types. The implicit sublanguage also allows $\etrm{M}$, an embedding of a
term of the explicit sublanguage inside a term of the implicit sublanguage.


Types $\hat{A}, \hat{B}, \hat{C}$ of the implicit sublanguage are not distinguished
as nullable or non-null. All types in the implicit sublanguage admit the null constant.

\subsection{Typing}

The typing rules of the two-language calculus extend those of the core calculus
from \Cref{fig:typing} for typing terms of the explicit sublanguage,
except the \rn{Blame} rule is removed, and the \rn{Cast} rule is restricted to subtypes,
and a new rule for embedded implicit terms is added, as shown in \Cref{fig:imptyping}.
\TODO{Should we remove the \rn{Blame} rule also from the core calculus?}
The typing rules for terms of the implicit sublanguage are also shown
in \Cref{fig:imptyping}; they are standard except for the \rn{Imp-Exp}
rule for explicit-sublanguage terms embedded in the implicit sublanguage.

\begin{figure}
    \infrule[Imp]
    {\typingiG{\hat{M}}{\hat{A}}}
    {\typingG{\itrm{\hat{M}}}{\typofityp{\hat{A}}}}

    \infrule[Imp-Var]
    {x:\hat{A} \in \Gamma}
    {\typingiG{\hat{x}}{\hat{A}}}

    \infax[Imp-Base]
    {\typingiG{\hat{c}}{\iota}}

    \infrule[Imp-Binop]
    {\typingiG{\hat{N}}{\iota} \andalso \typingiG{\hat{M}}{\iota}}
    {\typingiG{\binop{\hat{N}}{\hat{M}}}{\iota}}

    \infax[Imp-Null]
    {\typingiG{\hat{\code{null}}}{\hat{A}}}

    \infrule[Imp-Abs]
    {\typingi{\Gamma,x:\hat{A}}{\hat{N}}{\hat{B}}}
    {\typingiG{\abs{x}{\hat{A}}{\hat{N}}}{\arrow{\hat{A}}{\hat{B}}}}

    \infrule[Imp-App]
    {\typingiG{\hat{N}}{\arrow{\hat{A}}{\hat{B}}} \andalso \typingiG{\hat{M}}{\hat{A}}}
    {\typingiG{\app{\hat{N}}{\hat{M}}}{\hat{B}}}

    \infrule[Imp-Exp]
    {\typingG{M}{\typofityp{\hat{A}}}}
    {\typingiG{\etrm{M}}{\hat{A}}}

  \caption{Typing rules of null blame calculus for sublanguages with explicit and implicit nulls}\label{fig:imptyping}
\end{figure}


The embedding rules \rn{Imp} and \rn{Imp-Exp} type the embedded term
with an unmodified typing context $\Gamma$. This is in contrast to
\citet{DBLP:conf/ecoop/NietoRRL19}, where the typing rules for similar embedding
constructs require the embedded term to be typed in an empty context,
prohibiting an embedded term to refer to free variables defined outside
of it. To lift this restriction, we define typing contexts $\Gamma$
to contain two forms of bindings, bindings with explicit types
$x:A$ and bindings with implicit types $x:\hat{A}$. A variable
bound to an explicit (respectively implicit) type can only be used
in an explicit (respectively implicit) term, enforced by the
\rn{Var} (respectively \rn{Imp-Var}) rules. This prevents
substitution of an explicit term for a variable occurrence
in an implicit term and vice versa. However, for example,
if an explicit term is embedded inside an implicit term
that is embedded in another explicit term, the inner explicit
term may refer to variables defined in the outer explicit term,
which would not be allowed in the system of \citet{DBLP:conf/ecoop/NietoRRL19}.
\TODO{Say why this is important to model larger languages
    like Java and Scala.}

\subsection{Semantics}

We define the semantics of the two-language calculus as a translation
into the core calculus from \Cref{sec:defn}. The translation is
presented in \Cref{fig:translation}.

\begin{figure}
\textbf{Translation of Implict Types}\\
\begin{align*}
\typofityp{\iota} &= \iota?\\
\typofityp{\arrow{\hat{A}}{\hat{B}}} &= (\arrow{\typofityp{\hat{A}}}{\typofityp{\hat{B}}})?\\
\end{align*}
\textbf{Elvis Operator Syntactic Sugar}\\
\begin{align*}
\elvis{M}{p} &= \case{M}{\blame{p}}{x}{x}
\end{align*}
\textbf{Translation of Implict Terms}\\
\begin{align*}
\trmofitrm{\hat{x}} &= x\\
\trmofitrm{\hat{\code{null}}} &= \code{null}\\
\trmofitrm{\hat{c}} &= \lift{c}\\
\trmofitrm{\binop{\hat{N}}{\hat{M}}} &= \lift{\binop{\elvisP{\trmofitrm{\hat{N}}}{\blop}}{\elvisP{\trmofitrm{\hat{M}}}{\blop}}} \\
\trmofitrm{\abs{x}{\hat{A}}{\hat{N}}} &= \lift{\abs{x}{\typofityp{\hat{A}}}{\trmofitrm{\hat{N}}}} \\
\trmofitrm{\app{\hat{N}}{\hat{M}}} &= \app{\elvisP{\trmofitrm{\hat{N}}}{\blderef}}{\trmofitrm{\hat{M}}} \\
\trmofitrm{\etrm{M}} &= \trmofetrm{M}\\
\end{align*}
\textbf{Translation of Explicit Terms}\\
\begin{align*}
\trmofetrm{x} &= x\\
\trmofetrm{c} &= c\\
\trmofetrm{\binop{N}{M}} &= \binop{\trmofetrm{N}}{\trmofetrm{M}}\\
\trmofetrm{\code{null}} &= \code{null}\\
\trmofetrm{\lift{N}} &= \lift{\trmofetrm{N}}\\
\trmofetrm{\case{N}{M}{x}{L}} &= \case{\trmofetrm{N}}{\trmofetrm{M}}{x}{\trmofetrm{L}}\\
\trmofetrm{\abs{x}{A}{N}} &= \abs{x}{A}{\trmofetrm{N}}\\
\trmofetrm{\app{N}{M}} &= \app{\trmofetrm{N}}{\trmofetrm{M}}\\
\trmofetrm{\cast{M}{A}{p}{B}} &= \cast{\trmofetrm{M}}{A}{p}{B}\\
\trmofetrm{\blame{p}} &= \blame{p}\\
\trmofetrm{\itrm{\hat{M}}} &= \trmofitrm{\hat{M}} \\
\end{align*}
  \caption{Translation of null blame calculus for sublanguages with explicit and implicit nulls into the base null blame calculus}\label{fig:impsyntax}
\end{figure}


Types $\hat{A}, \hat{B}, \hat{C}$ from the implicit sublanguage
admit null, so they are translated to the corresponding nullable
types of the core calculus. Types $A,B,C$ from the explicit sublanguage
are also types of the core calculus, so no translation is needed.
\TODO{Consider changing the notation for types from the implicit
    language to write them \emph{always} with question marks: $\iota?$ and
    $(\arrow{A}{B})?$.
}

In the translation of terms of the implicit sublanguage, we will
make frequent use of pattern matching. We introduce as shorthand
an Elvis operator $\elvis{M}{p}$ that takes a term $M$ of nullable type $D?$,
and evaluates to $\blame{p}$ when $M$ evaluates to $\code{null}$
and to $V$ when $M$ evaluates to $\lift{V}$.

Variable references and the null constant are just translated to
themselves. Base constants $\hat{c}$, base operations $\binop{\hat{N}}{\hat{M}}$,
and function abstractions $\abs{x}{\hat{A}}{\hat{M}}$ have types
in the implicit language that translate to nullable types in the
explicit language, so these terms are translated to lifted terms
in the explicit language. The translation of a base operation $\oplus$
uses the Elvis operator to check the whether the operands
$\hat{N}$ and $\hat{M}$ are null before performing the opertaion $\oplus$.
The blame label $\blop$ is used to signal that a null check in a
base operation failed. Similarly, the translation of a function
application $\app{\hat{N}}{\hat{M}}$ first checks whether $\hat{N}$
(the function) evaluates to null before evaluating the argument
$\hat{M}$ and performing the application.

An explicit language term $M$ embedded as $\etrm{M}$ in the implicit
language is translated to $\trmofetrm{M}$, where $\trmofetrm{\cdot}$
is a translation that leaves explicit language terms mostly unchanged,
except to apply the implicit term translation function $\trmofitrm{\cdot}$
when the explicit term is an embedding $\itrm{\hat{M}}$ of an implicit term
$\hat{M}$.
