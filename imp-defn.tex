\section{An Implicitly Nullable Language\label{sec:impdefn}}

Having defined an explicit language with casts (like Scala), we now extend it 
with an implicit language that ignores nullability in its types (like Java).
Terms of the implicit language can be embedded within terms of the explicit language
and vice versa.
Previously, \citet{DBLP:conf/ecoop/NietoRRL19}
defined two new calculi with translations to a third core calculus.
In contrast, we reuse the explicit language from the previous sections directly,
and just add a new implicit language.

\subsection{Syntax}

The combined syntax of the explicit and implicit languages is defined in \Cref{fig:impsyntax}.
The terms $L, M, N$ of the explicit language are
those defined previously in \Cref{fig:syntax}, extended with $\itrm{\hat{M}}$, 
an embedding of a term $\hat{M}$ of the implicit language.
In general, we use a $\hat{\text{hat}}$ to mark elements of the implicit language.

\begin{figure}
  \textbf{Implicit Terms}\\
  \begin{align*}
      \hat{L}, \hat{M}, \hat{N} &\Coloneqq \ \hat{x}    \tag*{Variable} \\
      &\p \hat{c}                     \tag*{Base Constant} \\
&\p \binop{\hat{M}}{\hat{N}}          \tag*{Base Operation} \\
&\p \hat{\code{null}}           \tag*{Null Constant} \\
&\p \abs{x}{\hat{A}}{\hat{N}}         \tag*{Function Abstraction} \\
&\p \app{\hat{L}}{\hat{M}}            \tag*{Function Application} \\
  \end{align*}
  \textbf{Implicit Types}\\
  \begin{align*}
      \hat{A}, \hat{B}, \hat{C} &\Coloneqq \iota       \tag*{Base Type} \\
      &\p \arrow{\hat{A}}{\hat{B}}          \tag*{Function Type} \\
  \end{align*}
  \caption{Syntax of the language with implicit nulls}\label{fig:impsyntax}
\end{figure}


The syntax of terms of the implicit language $\hat{L}, \hat{M}, \hat{N}$
mirrors that of the explicit language, but omits lifting, pattern matching, casts, and blame,
since those are useless without the distinction between nullable and non-null
types. We also allow a term $\etrm{M}$ of the explicit language to be embedded
within a term of the implicit language.


Types $\hat{A}, \hat{B}, \hat{C}$ of the implicit language are not distinguished
as nullable or non-null. All types in the implicit language admit the null constant.
We define a translation of implicit language types to nullable types in the explicit language
as follows:
\begin{align*}
\typofityp{\iota} &= \iota?\\
\typofityp{\arrow{\hat{A}}{\hat{B}}} &= \arrowP{\typofityp{\hat{A}}}{\typofityp{\hat{B}}}?\\
\end{align*}
The translated types are equivalent in that they admit equivalent sets
of values: in particular, the translated types admit the null constant.

\TODO{Consider changing the notation for types from the implicit
    language to write them \emph{always} with question marks: $\iota?$ and
    $\arrowP{A}{B}?$.
}



\subsection{Typing}

The typing rules of the explicit language from \Cref{fig:typing} remain the
same, but one new rule \rn{Imp} is added for typing a term $\hat{M}$ of the implicit
language embedded in the explicit language, as shown in \Cref{fig:imptyping}.
Since a term of the explicit language can be an embedded term of the implicit
language and vice versa, we distinguish a second typing relation $\typingi{\cdot}{\cdot}{\cdot}$
for typing in the implicit language. Thus, the \rn{Imp} typing rule says that
if an implicit term $\hat{M}$ has type $\hat{A}$ in the implicit language,
then when it is embedded in the explicit language, it is considered in the
explicit language to have the type $\trmofitrm{\hat{A}}$.

The typing rules of the implicit sublanguage are also shown
in \Cref{fig:imptyping}. They are standard except for the \rn{Imp-Exp}
rule, which says that a term $M$ of the explicit language can be
embedded in the implicit language if its type $\trmofitrm{\hat{A}}$
in the explicit language is the image under the translation of some type $\hat{A}$ of
the implicit language.

\begin{figure}
    \infrule[Imp]
    {\typingiG{\hat{M}}{\hat{A}}}
    {\typingG{\itrm{\hat{M}}}{\typofityp{\hat{A}}}}

    \infrule[Imp-Var]
    {x:\hat{A} \in \Gamma}
    {\typingiG{\hat{x}}{\hat{A}}}

    \infax[Imp-Base]
    {\typingiG{\hat{c}}{\iota}}

    \infrule[Imp-Binop]
    {\typingiG{\hat{N}}{\iota} \andalso \typingiG{\hat{M}}{\iota}}
    {\typingiG{\binop{\hat{N}}{\hat{M}}}{\iota}}

    \infax[Imp-Null]
    {\typingiG{\hat{\code{null}}}{\hat{A}}}

    \infrule[Imp-Abs]
    {\typingi{\Gamma,x:\hat{A}}{\hat{N}}{\hat{B}}}
    {\typingiG{\abs{x}{\hat{A}}{\hat{N}}}{\arrow{\hat{A}}{\hat{B}}}}

    \infrule[Imp-App]
    {\typingiG{\hat{N}}{\arrow{\hat{A}}{\hat{B}}} \andalso \typingiG{\hat{M}}{\hat{A}}}
    {\typingiG{\app{\hat{N}}{\hat{M}}}{\hat{B}}}

    \infrule[Imp-Exp]
    {\typingG{M}{\typofityp{\hat{A}}}}
    {\typingiG{\etrm{M}}{\hat{A}}}

  \caption{Typing rules of null blame calculus for sublanguages with explicit and implicit nulls}\label{fig:imptyping}
\end{figure}


The embedding rules \rn{Imp} and \rn{Imp-Exp} type the embedded term
with an unmodified typing context $\Gamma$. This is in contrast to
\citet{DBLP:conf/ecoop/NietoRRL19}, where the typing rules for similar embedding
constructs require the embedded term to be typed in an empty context,
prohibiting an embedded term to refer to free variables defined outside
of it. To lift this restriction, we allow a typing context $\Gamma$
to contain two forms of bindings, bindings with explicit types
$x:A$ and bindings with implicit types $x:\hat{A}$. A variable
bound to an explicit (respectively implicit) type can only be used
in an explicit (respectively implicit) term, enforced by the
\rn{Var} (respectively \rn{Imp-Var}) rules. This prevents
substitution of an explicit term for a variable occurrence
in an implicit term and vice versa. However, for example,
if an explicit term is embedded inside an implicit term
that is embedded in another explicit term, the inner explicit
term may refer to variables defined in the outer explicit term,
which would not be allowed in the system of \citet{DBLP:conf/ecoop/NietoRRL19}.
\TODO{Say why this is important to model larger languages
    like Java and Scala.}

\subsection{Semantics}

We define the semantics of the implicit language by translation
to the explicit language, whose operational semantics we define in
\Cref{sec:defn}. The translation is presented in \Cref{fig:translation}.

\begin{figure}
\textbf{Translation of Implict Types}\\
\begin{align*}
\typofityp{\iota} &= \iota?\\
\typofityp{\arrow{\hat{A}}{\hat{B}}} &= (\arrow{\typofityp{\hat{A}}}{\typofityp{\hat{B}}})?\\
\end{align*}
\textbf{Elvis Operator Syntactic Sugar}\\
\begin{align*}
\elvis{M}{p} &= \case{M}{\blame{p}}{x}{x}
\end{align*}
\textbf{Translation of Implict Terms}\\
\begin{align*}
\trmofitrm{\hat{x}} &= x\\
\trmofitrm{\hat{\code{null}}} &= \code{null}\\
\trmofitrm{\hat{c}} &= \lift{c}\\
\trmofitrm{\binop{\hat{N}}{\hat{M}}} &= \lift{\binop{\elvisP{\trmofitrm{\hat{N}}}{\blop}}{\elvisP{\trmofitrm{\hat{M}}}{\blop}}} \\
\trmofitrm{\abs{x}{\hat{A}}{\hat{N}}} &= \lift{\abs{x}{\typofityp{\hat{A}}}{\trmofitrm{\hat{N}}}} \\
\trmofitrm{\app{\hat{N}}{\hat{M}}} &= \app{\elvisP{\trmofitrm{\hat{N}}}{\blderef}}{\trmofitrm{\hat{M}}} \\
\trmofitrm{\etrm{M}} &= \trmofetrm{M}\\
\end{align*}
\textbf{Translation of Explicit Terms}\\
\begin{align*}
\trmofetrm{x} &= x\\
\trmofetrm{c} &= c\\
\trmofetrm{\binop{N}{M}} &= \binop{\trmofetrm{N}}{\trmofetrm{M}}\\
\trmofetrm{\code{null}} &= \code{null}\\
\trmofetrm{\lift{N}} &= \lift{\trmofetrm{N}}\\
\trmofetrm{\case{N}{M}{x}{L}} &= \case{\trmofetrm{N}}{\trmofetrm{M}}{x}{\trmofetrm{L}}\\
\trmofetrm{\abs{x}{A}{N}} &= \abs{x}{A}{\trmofetrm{N}}\\
\trmofetrm{\app{N}{M}} &= \app{\trmofetrm{N}}{\trmofetrm{M}}\\
\trmofetrm{\cast{M}{A}{p}{B}} &= \cast{\trmofetrm{M}}{A}{p}{B}\\
\trmofetrm{\blame{p}} &= \blame{p}\\
\trmofetrm{\itrm{\hat{M}}} &= \trmofitrm{\hat{M}} \\
\end{align*}
  \caption{Translation of null blame calculus for sublanguages with explicit and implicit nulls into the base null blame calculus}\label{fig:impsyntax}
\end{figure}


In the translation of terms of the implicit sublanguage, we will
make frequent use of pattern matching. We introduce as shorthand
an Elvis operator $\elvis{M}{N}$ that takes a term $M$ of nullable type $D?$
and reduces to $N$ when $M$ evaluates to $\code{null}$
and to $V$ when $M$ evaluates to $\lift{V}$.

Variable references and the null constant are just translated to
themselves. Base constants $\hat{c}$, base operations $\binop{\hat{N}}{\hat{M}}$,
and function abstractions $\abs{x}{\hat{A}}{\hat{M}}$ have types
in the implicit language that translate to nullable types in the
explicit language, so these terms are translated to lifted terms
in the explicit language. The translation of a base operation $\oplus$
uses the Elvis operator to check the whether the operands
$\hat{N}$ and $\hat{M}$ are null before performing the opertaion $\oplus$.
The blame label $\blop$ is used to signal that a null check in a
base operation failed. Similarly, the translation of a function
application $\app{\hat{N}}{\hat{M}}$ first checks whether $\hat{N}$
(the function) evaluates to null before evaluating the argument
$\hat{M}$ and performing the application.
An explicit language term embedded in the implicit language is
just translated to itself, except that and implicit language terms
further embedded within the embedded explicit language term are transformed
using the translation.

% An explicit language term $M$ embedded as $\etrm{M}$ in the implicit
% language is translated to $\trmofetrm{M}$, where $\trmofetrm{\cdot}$
% is a translation that leaves explicit language terms mostly unchanged,
% except to apply the implicit term translation function $\trmofitrm{\cdot}$
% when the explicit term is an embedding $\itrm{\hat{M}}$ of an implicit term
% $\hat{M}$.
