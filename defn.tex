\section{The Explicit Language\label{sec:defn}}
In this section, we introduce a language that tracks the possibility of null references
explicitly in types. We will call it the explicit language for short.
We define the syntax, typing and subtyping rules, and a reduction relation.
In \Cref{sec:thm}, we will prove standard type safety and blame safety properties.

The syntax of the explicit language is shown in \Cref{fig:syntax}. The basic values
are constants $c$ of a base type $\iota$, function abstractions $\abs{x}{A}{N}$
of a function type $\arrow{A}{B}$, and the null constant $\code{null}$.
In addition to the two \emph{definite types} $\iota$ and $\arrow{A}{B}$,
the type system includes \emph{nullable types} $D?$, where $D$ is any definite type.
The constructors of $D?$ are the null constant $\code{null}$ and the
lift operation $\lift{N}$, where $N$ is a term of type $D$.
When $V$ is a value, $\lift{V}$ is also considered a value.
A cast $\cast{V}{\arrow{A}{B}}{p}{\arrow{A'}{B'}}$ of a function value $V$ is 
also a value. When such a cast-function value is applied to an argument, the
argument will first be cast from $A'$ to $A$, then the function $V$ will be
applied to it, and finally the result will be cast from $B$ to $B'$.

In addition to values, the calculus includes terms for function application $\app{L}{M}$,
general casts $\cast{M}{A}{p}{B}$, a pattern matching construct $\case{L}{M}{x}{N}$ that
destructs terms of nullable types $D?$, and a failure result $\blame{p}$. As is standard
in gradual type systems, each cast has a blame label $p$ so that the result of a failing computation
can be traced to the cast that failed. A blame label can be positive $p$, indicating
that the term inside the cast caused the cast to fail, or negative $\overline{p}$,
indicating that the context in which the cast appears caused the cast to fail.

\begin{figure}
  \textbf{Labels and Variables}\\
  \begin{align*}
      x    &                             \tag*{Variables} \\
      p, q, \overline{p}, \overline{q}    &                             \tag*{Blame Labels} \\
  \end{align*}
  \textbf{Terms}\\
  \begin{align*}
L, M, N &\Coloneqq \ x    \tag*{Variable} \\
&\p c                     \tag*{Base Constant} \\
&\p \binop{M}{N}          \tag*{Base Operation} \\
&\p \abs{x}{A}{N}         \tag*{Function Abstraction} \\
&\p \app{L}{M}            \tag*{Function Application} \\
&\p \code{null}           \tag*{Null Constant} \\
&\p \lift{M}              \tag*{Lift} \\
&\p \case{L}{M}{x}{N}     \tag*{Case} \\
&\p \cast{M}{A}{p}{B}     \tag*{Cast} \\
&\p \blame{p}             \tag*{Blame} \\
  \end{align*}
  \textbf{Values}\\
  \begin{align*}
V, W &\Coloneqq c                     \tag*{Base Constant} \\
&\p \binop{V}{W}          \tag*{Base Operation on Values} \\
&\p \abs{x}{A}{N}         \tag*{Function Abstraction} \\
&\p \code{null}           \tag*{Null Constant} \\
&\p \lift{V}              \tag*{Lift of a Value} \\
&\p \cast{V}{\arrow{A}{B}}{p}{\arrow{A'}{B'}}     \tag*{Function-typed Cast of a Value} \\
  \end{align*}
  \textbf{Types}\\
  \begin{align*}
A, B, C &\Coloneqq \ D    \tag*{Definite Type} \\
&\p D?                    \tag*{Nullable Type} \\
\\
D, E  &\Coloneqq \iota       \tag*{Base Type} \\
&\p \arrow{A}{B}          \tag*{Function Type} \\
  \end{align*}
  \caption{Syntax of null blame calculus}\label{fig:syntax}
\end{figure}


The typing rules of the explicit language are shown in \Cref{fig:typing}.
The rules for variables, base type constants and operations, and function
abstraction and application are standard. The
\rn{Null} and \rn{Lift} rules identify the null constant $\code{null}$ and the lift
operation $\lift{M}$ as the constructors of a nullable type $D?$.
The \rn{Blame} rule specifies that a failure result $\blame{p}$ is possible at
any type $A$. The \rn{Cast} rule allows casts from type $A$ to type $B$ as
long as $A$ and $B$ are \emph{compatible}, written $\arrow{A}{B}$. Informally, two types are compatible
if they have the same structure, but differ only in the nullability of their components.
Finally, the \rn{Case} rule specifies that the $\code{case}$ construct
destructs terms of a nullable type $D?$.

\begin{figure}
    \infrule[Var]
    {x:A \in \Gamma}
    {\typingG{x}{A}}

    \infax[Base]
    {\typingG{c}{\iota}}

    \infrule[Binop]
    {\typingG{N}{\iota} \andalso \typingG{M}{\iota}}
    {\typingG{\binop{N}{M}}{\iota}}

    \infax[Null]
    {\typingG{\code{null}}{D?}}

    \infrule[Lift]
    {\typingG{N}{D}}
    {\typingG{\lift{N}}{D?}}

    \infrule[Abs]
    {\typing{\Gamma,x:A}{N}{B}}
    {\typingG{\abs{x}{A}{N}}{\arrow{A}{B}}}

    \infrule[App]
    {\typingG{N}{\arrow{A}{B}} \andalso \typingG{M}{A}}
    {\typingG{\app{N}{M}}{B}}

    \infax[Blame]
    {\typingG{\blame{p}}{A}}

    \infrule[Cast]
    {\typingG{M}{A} \andalso \compat{A}{B}}
    {\typingG{\castP{M}{A}{p}{B}}{B}}

    \infrule[Case]
    {\typingG{N}{D?} \andalso \typingG{M}{A} \andalso \typing{G,x:D}{L}{A}}
    {\typingG{\case{N}{M}{x}{L}}{A}}

    \infax[Compat-Base]
    {\compat{\iota}{\iota}}

    \infrule[Compat-Null-R]
    {\compat{A}{D}}
    {\compat{A}{D?}}

    \infrule[Compat-Null-L]
    {\compat{D}{A}}
    {\compat{D?}{A}}

    \infrule[Compat-Arrow]
    {\compat{A}{A'} \andalso \compat{B}{B'}}
    {\compat{\arrow{A}{B}}{\arrow{A'}{B'}}}

  \caption{Typing rules of null blame calculus}\label{fig:typing}
\end{figure}


The operational semantics of the explicit language is shown in \Cref{fig:reduction}.
The \rn{App} rule is standard $\beta$-reduction. 
The \rn{Cast-App} rule defines $\beta$-reduction for a function wrapped
in a cast, ensuring that that the argument $W$ and the final result of the
function application are cast accordingly.
There are four rules for reducing casts from a nullable type $D?$.
A cast of $\code{null}$ to another nullable type $E?$ reduces to just $\code{null}$ (\rn{Cast-Null-Nullable}). 
A cast of $\code{null}$ to a non-nullable type $E$ reduces to $\blame{p}$ (\rn{Cast-Null-NonNull}).
A cast of a lifted value $\lift{V}$ from type $D?$ to a non-nullable
type $E$ evaluates to $V$ wrapped in a cast from $D$ to $E$ (\rn{Cast-Lift-NonNull}).
When such a lifted value is cast to a \emph{nullable} type $E?$,
this result is additionally lifted: $\lift{\cast{V}{D}{p}{E}}$ (\rn{Cast-Lift-Nullable}).
A cast from a base type can only be back to the base type;
it reduces to the value $V$ inside the cast (\rn{Cast-Base}).
Two rules reduce the pattern-matching $\code{case}$ construct.
When the scrutinee is $\code{null}$, the $\code{case}$ reduces
to the term in the $\code{null}$ branch (\rn{Case-Null}).
When the scrutinee is a lifted value $\lift{V}$, the
$\code{case}$ reduces to the term in the non-null branch,
with $V$ substituted for the parameter $x$ (\rn{Case-NonNull}).
A grammar of evaluation contexts $\mathcal{E}$ ensures call-by-value reduction
in function applications, inside casts, pattern matches, lift operations,
and base type operations. The \rn{Ctx} rule specifies reduction inside
an evaluation context. The \rn{Err} rule specifies that a cast failure
inside an evaluation context floats up to the top level, terminating
the reduction sequence.

\begin{figure}
\infax[App]
{\red{(\abs{x}{A}{N})V}{\subst{N}{x}{V}}}

\infax[Cast-App]
{\red{\app{\castP{V}{\arrow{A}{B}}{p}{\arrow{A'}{B'}}}{W}}
     {\cast{\appP{V}{\castP{W}{A'}{\overline{p}}{A}}}{B}{p}{B'}}}

\infax[Cast-Null-Nullable]
{\red{\cast{\code{null}}{D?}{p}{E?}}{\code{null}}}

\infax[Cast-Lift-Nullable]
{\red{\cast{\lift{V}}{D?}{p}{E?}}{\lift{\cast{V}{D}{p}{E}}}}

\infax[Cast-Null-NonNull]
{\red{\cast{\lift{V}}{D?}{p}{E}}{\blame{p}}}

\infax[Cast-Lift-NonNull]
{\red{\cast{\lift{V}}{D?}{p}{E}}{\cast{V}{D}{p}{E}}}

\infax[Cast-Base]
{\red{\cast{V}{\iota}{p}{\iota}}{V}}

\infax[Case-Null]
{\red{\case{\code{null}}{M}{x}{L}}{M}}

\infax[Case-NonNull]
{\red{\case{\lift{V}}{M}{x}{L}}{\subst{L}{x}{V}}}

\infrule[Ctx]
{\red{N}{M}}
{\red{E[N]}{E[M]}}

\infax[Err]
{\red{E[\blame{p}]}{\blame{p}}}

%\infrule[App-CtxL]
%{\red{M}{M'}}
%{\red{\app{M}{N}}{\app{M'}{N}}}
%
%\infrule[App-CtxR]
%{\red{N}{N'}}
%{\red{\app{V}{N}}{\app{V}{N'}}}
%
%\infax[App-ErrL]
%{\red{\app{(\blame{p})}{N}}{\blame{p}}}
%
%\infax[App-ErrR]
%{\red{\app{V}{(\blame{p})}}{\blame{p}}}

$$ E \Coloneqq [] \p \app{E}{N} \p \app{V}{E} \p \cast{E}{A}{p}{B} \p \case{E}{M}{x}{L} \p \lift{E} \p \binop{E}{M} \p \binop{V}{E}$$
  \caption{Reduction rules of null blame calculus}\label{fig:reduction}
\end{figure}


